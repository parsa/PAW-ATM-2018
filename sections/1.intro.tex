\section{Introduction}
Global address space systems attempt to boost productivity and simplify the
application development cycle of distributed applications by providing a
ubiquitous abstraction layer over memory spaces that are provided and managed
by the operating system on each computer. SPMD-style (Single Program Multiple
Data) Partitioned Global Address Space designs eliminate this layer during
compilation to avoid the complexity of resolving global addresses at runtime at
the cost of limiting productivity and imposing limitations on the code, while
others like HPX\cite{Kaiser2014,hpx_repo}, HPX5\cite{hpx5},
Charm++\cite{Kale1993}, UPC++\cite{Zheng2014}, and Chapel\cite{chapel_lang} provide a
runtime component that maps global addresses to virtual addresses during
application execution to provide true global addresses.


The demand for models that enable applications to process massive
datasets within specific time, power, and budgetary constraints continues to
pose challenges for the computer science community
\cite{Amarasinghe091exascale,Sterling2009}. One category of such complexities
includes managing a large quantity of objects across several machines and
memory partitions while maximizing data locality. Several Partitioned Global
Address Space systems (PGAS) \cite{pgasorg} try to address these needs by
providing control over data distribution and facilitating data accesses across
processes and machines. However, initial data placement alone is not sufficient
to address more complex issues such as load balancing applications that run on heterogeneous clusters or applications
that become scaling impaired over time due to increasing load imbalance such
as adaptive mesh refinement (AMR), dynamic graph applications, and partial
differential equation (PDE) solvers\cite{5364511,Anderson2011a,Dekate2011}.
Active Global Address Space (AGAS) is a system that tries to address
performance impaired applications. AGAS was initially proposed for the ParalleX
programming model\cite{5364511} and implemented in HPX. It adds an abstraction
layer on top of local objects on each computing node by mapping local virtual
addresses to a global address and ensuring that global addresses are valid even
if the object it refers to is migrated to a different physical location. This
abstraction layer needs to execute code to resolve and maintain the references
and takes a portion of the execution time.


Assessing the overheads caused by AGAS is the main objective of this paper. We
address this objective by developing a system that uses measurement data from
AGAS and applies it to identify AGAS functions that exhibit poor scaling
behavior in HPX 0.9. In the rest of this work, we discuss other pertinent
research in section \ref{related_work}, present a general overview of AGAS
functionality in section \ref{agas}, explain the criteria based on which the
results can be evaluated in section \ref{performance}, experiments that were
run and examine their results in section \ref{results}, and further analyze
them and consider directions that are likely to produce more insights into
improving AGAS considering our findings in section \ref{conclusions}.
