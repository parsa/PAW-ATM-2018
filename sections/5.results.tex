\section{Performance Results}
\label{results}

As mentioned before, the main objective of this study is to identify the AGAS
operations that need performance improvement and to study the effects of strong
scaling on AGAS. In this section, we present the results of this study.

\begin{figure}[h]
    \centering
    \vspace{2.5cm}
    \caption{Total time spent to resolve GID calls while running OctoTiger on 2 to 200 nodes of SuperMIC with fixed problem size (Strong Scaling). Each (red) circle refers to the measured amount of time a locality has spent executing resolve GID calls. Higher intensities indicate overlapping measurements. The line highlights locality 0's behavior.}
    \includegraphics[width=.67\textwidth,height=\textheight,keepaspectratio]{graphs/octotiger_resolve_gid_time}
    \label{fig:octgr_strong_resolve_gid_time}
\end{figure}

Appropriate distribution of work is of particular interest to computer
scientists. In case of AGAS, ideally we expect the number of specific
operations and the amount of time spent to perform them to be similar. Address
resolution is one of the main functions of AGAS that translates a HPX GID to
the actual virtual address on a machine. Its operation is complicated if the
object is not currently located on the locality address resolution is taking
place and in such case AGAS tries to determine which locality the object
currently lives on, serialize the task that asked to access the foreign GID in
an HPX parcel, and send it to the locality object is living in. If an AGAS
instance has no knowledge of the locality that is holding an object then it
takes advantage of the fact that a locality an object is created on stays
responsible for it during the object's entire lifetime and uses the metadata in
the GID to determine the locality the object was originially created on and
sends the query there. Fig. \ref{fig:octgr_strong_resolve_gid_time} shows the
amount of time GID resolution takes while running and OctoTiger. By looking at
the graph we notice that while the the amount of work performed by each
locality is relatively similar up to 146 nodes, locality 0 starts to perform
more work that other localities. This may indicate that most objects globally
referenced were created on locality 0 and it requires further investigation to
determine if this effect is caused by application behavior, non-optimal data
distribution, or HPX itself.
% What else do I say?!

Parcels are the basic communication block in HPX. Parcels are active messages
that trigger an operation on the target AGAS instance that opens them and
usually contain a serialized task and its arguments. Parcel routing is the
operation in which the a parcel is sent to a different locality. Fig.
\ref{fig:octgr_strong_route_time} shows that locality 0 ends up handling a
disproportionate and increasing number of routing requests regardless of the
application profile and problem size and hence, designate the routing operation
as candidate for optimization.

\begin{figure}[h]
    \centering
    \vspace{2.5cm}
    \caption{Total time spent in route calls while running OctoTiger on 2 to 200 nodes of SuperMIC with fixed problem size (Strong Scaling). Each (red) circle refers to measured amount of time a locality has spent executing resolve GID calls. Higher intensities indicate overlapping measurements. The line highlights locality 0's behavior.}
    \includegraphics[width=.67\textwidth,height=\textheight,keepaspectratio]{graphs/octotiger_route_time}
    \label{fig:octgr_strong_route_time}
\end{figure}
% Garbage collection not done much in OctoTiger
For a garbage collection system to function the HPX runtime system must be able
to determine if an object is currently being used. HPX uses reference counting
for local references and a credit-based system to keep track of global
references. The operations that lend and return credits are called decrement
credit and increment credit, respectively.
%One issue that affects our
%experiment is that OctoTiger uses unmanaged objects and therefore AGAS garbage
%collection is not heavily relied upon and therefore the amount of time spent on
%global reference management is not a fair example of applications that do use
%garbage collection.  Despite this limitation, the large number of nodes and the
%limited usage of the operation still enables us to observe the changes in
%decrement credit's behavior as the number of nodes increase.
Fig \ref{fig:octgr_strong_decr_cred} depicts the behavior of decrement\_credit
calls taken place while running OctoTiger with the same number of objects as
the number of nodes are increased. This figure shows that a
linear increase in number of objects results in an exponential growth
in the amount of time spent on managing global references. Although this
behavior is likely to be different for applications that do not have as many
global references per each object and therefore decrement\_credit is a good
candidate for further optimizations.

\begin{figure}[h]
    \centering
    \vspace{2.5cm}
    \caption{Total time spent in decrement calls while running OctoTiger on 2 to 200 nodes of SuperMIC with fixed problem size (Strong Scaling). Each (red) circle refers to measured amount of time a locality has spent executing resolve GID calls. Higher intensities indicate overlapping measurements. The line highlights locality 0's behavior.}
    \includegraphics[width=.67\textwidth,height=\textheight,keepaspectratio]{graphs/octotiger_decrement_credit_time}
    \label{fig:octgr_strong_decr_cred}
\end{figure}

%\begin{figure}
%    \begin{subfigure}[b]{0.3\textwidth}
%        \centering
%        \caption{Total time spent in resolve GID calls while running OctoTiger on 2 to 200 nodes of SuperMIC with fixed problem size (Strong Scaling)}
%        \includegraphics[width=\textwidth,keepaspectratio]{graphs/octotiger_resolve_gid_time}
%        \label{fig:octgr_strong_resolve_gid_time}
%    \end{subfigure}%
%    \hfill
%    \begin{subfigure}[b]{0.3\textwidth}
%        \caption{Total time spent in route calls while running OctoTiger on 2 to 200 nodes of SuperMIC with fixed problem size (Strong Scaling)}
%        \includegraphics[width=\textwidth,height=\textheight,keepaspectratio]{graphs/octotiger_route_time}
%        \label{fig:octgr_strong_route_time}
%    \end{subfigure}%
%    \hfill
%    \begin{subfigure}[b]{0.3\textwidth}
%        \caption{Total time spent in decrement calls while running OctoTiger on 2 to 200 nodes of SuperMIC with fixed problem size (Strong Scaling)}
%        \includegraphics[width=\textwidth,height=\textheight,keepaspectratio]{graphs/octotiger_decrement_credit_time}
%        \label{fig:octgr_strong_decr_cred}
%    \end{subfigure}%
%\end{figure}

