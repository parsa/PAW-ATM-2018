\section{Quantifying AGAS Performance}
\label{performance}

The main challenge of any HPC global addressing system, including AGAS, is to
efficiently support the massive amounts of data that applications use during
execution. Our aim here, therefore, is to understand the efficiency of AGAS
through measuring its performance of associated overheads.
To study AGAS, we use OctoTiger\cite{kadam2017numerical,heller2017harnessing,octotiger_repo} as our
application and run it on the cluster provided by the Center of Computation and
Technology at Louisiana State University. It has to be noted that any other
application benchmark can be used in place of OctoTiger. However, developing an
HPX application takes work and at this point, OctoTiger is the only
available real world HPX application. In the rest of this section we present
OctoTiger and the performance metrics we use to measure the overheads
of AGAS. 

\subsection{OctoTiger}
%\todo[inline, color=red!50]{What is it? Why did we choose it? What are the implications of this choice?}%
LSU OctoTiger is a state of the art multi-physics AMR HPX application that
simulates the merger of double white dwarf binaries. Its computations include
hydrodynamics, gravitational, and radiation transportation solvers. It fully
utilizes the capabilities of AGAS in HPX since it dynamically changes the
computational resolution based on the actual needs of the simulation and thus, 
it is an inherently balance impaired application. It is also a memory intensive
application compared to most exascale challenge problems. For example,
Quicksilver and Pennant from CORAL2 benchmarks have a high water mark of 16\%
and 8\%, respectively, whereas for OctoTiger this number is about 60\%.

We use OctoTiger to study the behavior of AGAS when used by applications
running on large machines. We ran our experiments on SuperMIC at Louisiana
State University, a hybrid cluster composed of Xeon, Xeon Phi Knights Corner, 
and NVIDIA Tesla processors. Due to the complexity of the computations
performed by OctoTiger, adjusting the size of the problem with reasonable
accuracy to demonstrate weak scaling is not practical and therefore we
presenting the impacts of strong scaling on AGAS behavior.

\subsection{System and Environment Setup}
%\todo[inline, color=red!50]{Where are the experiments run? Why? What are the implications of this choice?}%
We run our experiments on SuperMIC, a hybrid Xeon/Xeon Phi cluster that
comprises 360 compute nodes located at Louisiana State University. More
details about the configuration of SuperMIC is included in Table
\ref{tab:machines}. We run OctoTiger on SuperMIC from
two nodes to 194 nodes only using the Xeon processors on each machine using
HPX 0.99\cite{hartmut_kaiser_2015_33656}. The software configuration is listed in Table  \ref{tab:software-used}

% Sample LLNCS table
%\begin{table}
%    \caption{SuperMIC Configuration}
%    \label{tab:machines}
%    \begin{center}
%    \begin{tabular}{r@{\quad}rl}
%    \hline
%    \multicolumn{1}{l}{\rule{0pt}{12pt}
%                       Year}&\multicolumn{2}{l}{World population}\\[2pt]
%    \hline\rule{0pt}{12pt}
%    8000 B.C.  &     5,000,000& \\
%      50 A.D.  &   200,000,000& \\
%    1650 A.D.  &   500,000,000& \\
%    1945 A.D.  & 2,300,000,000& \\
%    1980 A.D.  & 4,400,000,000& \\[2pt]
%    \hline
%    \end{tabular}
%    \end{center}
%\end{table}

\begin{table}
    \caption{SuperMIC Configuration}
    \centering
    \label{tab:machines}
    \begin{tabular}{cl}  
        \toprule
        Nodes available  & 360                             \\
        Architecture     & Ivy Bridge,                     \\
                         & Knights Corner                  \\
        Cores/Node       & $2 \times 10$ cores             \\
        Processor        & Intel Xeon E5-2680 @ 2.8 GHz,   \\
                         & Intel Xeon Phi 7120P @1.238 GHz \\
        Memory           & 64 GB DDR3 @ 1,866 MHz          \\
                         &                                 \\
        Connection       & 56 Gbps Infinitband             \\
        Operating System & Red Hat Enterprise Linux 6.8    \\
        \bottomrule
    \end{tabular}
\end{table}

\begin{table}
    \caption{Software used in the experiment}
    \centering
    \label{tab:software-used}
    \begin{tabular}{cl}  
        \toprule
        Compiler         & GCC 4.9.0                       \\
        MPI              & MVAPICH 2.0                     \\
        Boost            & 1.61.0                          \\
        hwloc            & 1.10                            \\
        HPX              & 0.99                            \\
        \bottomrule
    \end{tabular}
\end{table}

\begin{itemize}
\item{Performance Metrics}:
Every globally accessible object in HPX has a global ID that AGAS manages and
resolves to local virtual addresses at runtime. However, address resolution at runtime is
an overhead that consumes computational resources that application developers expect to be used by the
applications rather than the runtime system. To quantify and study the overhead of AGAS, we look at the following
fundamental operations, the number of calls, and the amount of time that is
spent performing these operations.

\item{Bind, Unbind, Object Lookup}:
Bind and Unbind operations occur when a global object is created and deleted, 
respectively. An object lookup operation takes place each time AGAS attempts 
to resolve a global ID.

\item{Locality Lookup}:
Each AGAS instance only knows about itself and locality 0. When an AGAS
instance needs to communicate with another locality and does not have
information about the appropriate communication endpoint to do so, it needs to
query that information from locality 0.

\item{Parcel Routing}:
HPX implements active messages in the form of parcels\cite{wagle2018methodology}. A parcel is packaged
information that triggers an operation upon reception by an AGAS instance. For
example, when an HPX task needs to operate on information that resides on
another locality, AGAS serializes the task along with its arguments and state
information and forwards it to the locality on which data resides.

\item{Cache}:
Whenever AGAS decides that an address is likely to be queried again, it stores
that information in the local AGAS cache of that locality. To serve its purpose of actually improving
performance the AGAS cache is designed to hold a limited number of entries that is
defined by user configuration.

\item{Garbage Collection}:
\label{sec:garbage_collection}
HPX provides managed objects whose lifetime is controlled by the garbage
collection mechanism instead of the programmer having to free the memory
consumed by each object. HPX uses reference counting to track references local
to each AGAS instance and a credit scheme to manage global references. Once an
object no longer has any local references and/or all its global reference
credits are returned, it is deemed garbage and is collected when the garbage
collection operation is triggered by reaching a certain threshold or by the
application developer in their code. Registration and removal of global references is done by
calling \textit{increment\_credit} and \textit{decrement\_credit} functions, respectively.
\end{itemize}
