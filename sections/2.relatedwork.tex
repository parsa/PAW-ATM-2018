\section{Related Work}
\label{related_work}

\todo{You need a section on background of global address spaces, why they are needed, how do they work? Not everyone knows this.}

In recent years, there has been a significant increase in utilization of
heterogeneous clusters that use GPUs and MICs in addition to
CPUs\cite{Lena2014,Yang2011266,Potluri2014,Sidelnik2011}. One approach to
manage such systems is using
solutions\cite{Rabenseifner2009,Yang2011266,Chorley2010} that separately use a
library like MPI for explicit communication between nodes and a choice of
shared-memory programming framework such as OpenMP\cite{openmp_org}, Kokkos\cite{kokkos_paper,kokkos_repo}, or
UPC\cite{upc_org}. Other approaches include Asynchronous PGAS
runtimes\cite{Saraswat2010} and Charm++\cite{charm_edu} that use dynamic
multithreading to avoid fragmenting the application development process to
separately manage communication and computation while maintaining portability
between various cluster configurations and providing access to heterogeneous
computing resources\cite{P0234R0}.

Most studies on distributed runtime systems do not include quantitative
analysis of the performance of their global address space system but present
the overall performance of applications using the respective model or
implementation. However, amongst the runtime systems mentioned only HPX has a
ubiquitous address system that is present during run time. This unique property
calls for a closer look into HPX's Active Global Address Space system behavior
and performance and is the motivation behind this study.
